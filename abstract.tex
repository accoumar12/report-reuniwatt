\section*{Abstract}
\addcontentsline{toc}{section}{Abstract}

For day-ahead forecasting, the combination of Numerical Weather Prediction (NWP) models and post-processing algorithms is the most popular method.

However, it is hard to extract from all the literature on the subject the best algorithm to use because of the lack of consistency in the different approaches.
Indeed different works use different datasets, metrics and even cross-validation methods. 

During my internship, my mission was to investigate the best algorithms according to the literature so as to improve the day-ahead irradiance forecasts. 
The comparison is initially conducted on 4 sites over the years 2021 and 2022, and aims at post-processing GFS forecasts data so as to lower both the mean average error (MAE) and the root mean square error (RMSE).
I eventually studied the hybridation of different forecast NWP models (AROME, GFS, ARPEGE, ECMWF), on 25 more validation sites, in order to benchmark it against the current algorithm used by Reuniwatt.

My final results demonstrated the benefits of post-processing both a single NWP model and several NWP models altogether.
While a linear model perform good enough for a RMSE optimisation, a support vector regression (SVR) model proved its ability to lower the MAE.
The ultimate model showcased improved metrics with respect to the current algorithm used by Reuniwatt.
