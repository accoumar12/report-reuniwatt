\section{Methodology}
\subsection{Data source}
\cite{verbois_statistical_2022} demonstrated that using a large set of predictors can significantly improve the performances of post-processing models, while \cite{suksamosorn_post-processing_2021}
selected WRF forecasts of irradiance, temperature, relative humidity and the solar zenith angleas relevant inputs of the models.

The forecasted data is for each day the one relative to the origin 00:00 UTC of the day before.
Our initial data source for the forecasts was GFS, and we opted for the following set of predictors, easily available for any location:

\begin{table}[h]
    \centering
    \begin{tabularx}{\textwidth} { 
  | >{\centering\arraybackslash}X 
  | >{\centering\arraybackslash}X 
  | >{\centering\arraybackslash}X 
  | >{\centering\arraybackslash}X
  | >{\centering\arraybackslash}X 
  |}
 \hline
 $ghi_{GFS}$ & $T_{GFS}^{2m}$ & $\theta$ & $\phi$ & $ghi_{cs}$ \\
 \hline
 \scriptsize Irradiance forecasted  & \scriptsize Temperature forecasted 2 meters above the ground & \scriptsize Zenith angle & \scriptsize Azimuth angle & \scriptsize Clear-sky irradiance \\
\hline
\end{tabularx}
    \caption{Set of predictors.}
    \label{tab:set_pred}
\end{table}

\cite{verbois_statistical_2022} advises researchers to analyze theirs models' performances over several years but I was at this point limited by the Reuniwatt API, thus
I opted initially for learning during 2020 and testing during 2021.
The four inital 
\subsection{Metrics}
\subsection{Models investigated}
\subsection{Model performances evaluation strategy}